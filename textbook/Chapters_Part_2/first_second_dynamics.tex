        Roughly stated, differential equations is the study of a system that undergoes change.  This change can depend on time, space, or both.  The history of differential equations begins with Newton's study of classical mechanics.  Newton was investigating the motion of objects in space and derived the first examples of differential equations.  The field quickly grew with the once other scientists noticed the wide scope of applicability of differential equations.

        In the time of Newton's classical mechanics, we saw the advent of celestial, fluid, and continuum mechanics. Examples include the diffusion, advection, and wave equations.  Other fields of science began to use these ideas from physics to model population dynamics and chemical reaction.  As time moved forward, more complicated physical interactions brought even more uses of differential equations to the forefront.  These were the dyanamical theories of electromagnetism and thermodynamics.  All of this happened before the turn of the 20$^\textrm{th}$ century.

        As we moved into the 1900s, there was the boom of modern physics with work from Einstein and many other scientists.  Einstein described the motion of atoms in a probabilistic manner that was also deeply related to differential equations of classical mechanics.  Soon after, he then  considered how spacetime itself behaves as a coupled dynamical continuum, much like the head of a drum that vibrates.  It was shortly after this discovery that Sch\"odinger and Heisenberg independently developed the theories of quantum mechanics.  Both stated the problem in different (but equivalent) ways.  This study brought together the notion of motion of particles with that of waves.

        Time passed, and in the mid 1900s computers were developed.  This forever changed the study of differential equations.  The problem was, we were finding that (other than very specific nice examples) most differential equations were extremely hard to solve.  In fact, many are so wild that the dynamics we see is volatile to the point of the so called ``butterfly effect."  These volatile systems are known as chaotic and are abundant in nature.  Weather, population, and chemical reaction are all areas where chaos can show up.  However, the ability to approximate solutions with computers allows us to make reasonable predictions and essentially solve problems that were previously deemed impossible.

        The goal for us is not to learn to solve many differential equations with a handbook of techniques.  These techniques can be readily found online and are very formulaic.  If you do them once, you can do them again.  Instead, our goal is to understand what differential equations model and what they say about systems.  Of course, we will explicitly solve some and see a few techniques, but that is not the emphasis.  If one pursues mathematical modelling, one will almost surely be working with computers to solve problems rather than by hand. It is with this mentality that we carry on to uncovering this structure of differential equations.

        \section{Ordinary Differential Equations}

        The first stop on the study of differential equations are the Ordinary Differential Equations \index{ordinary differential equation} (ODEs).  ODEs are equations that involve a single independent variable $t$ that we usually think of as time, a function (or dependent variable) $x(t)$, and derivatives of the function $x'(t)$, $x''(t)$, $x'''(t)$, up to $n$ derivatives $x^{(n)}(t)$.  We will not worry ourselves with the higher order equations yet.

        Unlike previous problems where we solve for a variable, or compute derivatives, we wish to find a function that satisfies the differential equation.  So, our aim is to find $x(t)$ given our understanding of how $x$ changes over time.

        \begin{ex}{First Order ODE}{first_order}
        Suppose we are asked to find a function $x(t)$ that satisfies the following Ordinary Differential Equation (ODE):
        \[
        x'(t) = f(t,x(t)),
        \]
        or with reduced notation
        \[
        x'=f(t,x).
        \]
        This is an example of the most general \emph{first order} ODE.
        \begin{itemize}
            \item Written in English, this equation says, ``what function $x(t)$ has a derivative that is equal to a function of $t$ and $x(t)$?"
            \item The function $f(t,x)$ is a function of two variables.
        \end{itemize}
        \end{ex}

        \noindent Often we will suppress some notation and let $x(t)$ be denoted by $x$.  We cannot forget that $x$ is a function of $t$, but it will be notationally more convenient to make this substitution. We should also say what a function of two variables is briefly.

        For example, we can take a function $f(t,x)$ and specify what it does with input values $t$ and $x$ simultaneously. A specific example could be
        \[
        f(t,x)=tx,
        \]
        or
        \[
        g(t,x)= t\sin\left(x^2\right),
        \]
        or
        \[
        h(t,x)= e^{t}\sqrt{x^2+tx+t^2}.
        \]

        \begin{exercise}
        Write your own function $p(t,x)$.
        \end{exercise}

        If we let the right hand side be given by $f(t)$, we arrive at the most fundamental example of a differential equation. In fact, this is an equation that you have not only already seen, but it is one that you know how to solve.  Indeed, consider the expression
        \[
            x'(t) = f(t)
        \]
        Here we see that we know the derivative of some function $x(t)$ is given by another function of $t$.  If the desire is to determine an expression for $x(t)$, then we can solve these equations simply through integration. That is,
        \[
            x(t) = \int f(t) dt.
        \]
        Thus, $x(t)$ is simply the antiderivative of $f(t)$ in this case.  Notice, that we will have a constant of integration in this expression.

        However, not all differential equations come out this nicely. The techniques needed to solve more complicated expressions are going to be developed in the succeeding sections.  Our attempts will be to classify the types of equations and find methods to find solutions for those types of equations. In general, we will stick to the most nicely behaved equations that appear throughout the natural world.


        \begin{df}{Order of an ODE}{order}
        The \boldgreen{order} \index{order!differential equation} of an ODE is the highest derivative that appears in the ODE.
        \end{df}

        \noindent Though right now we will not investigate higher order ODEs, we should at least know what order means.  Keep this in mind as we progress.  It turns out (though we don't necessarily get to it in this course) that higher order ODEs will be equivalent to many first order ODEs.

        \begin{ex}{Exponential Growth and Decay}{exp_growth_decay}
        As opposed to a very general set up, let us consider the following problem statement.\\

        \emph{The concentration of Plutonium in a vessel is measured over time.  It's found that the rate of change of this concentration is proportional to the current concentration.  What ODE models this situation?}\\

        The answer to the above question is
        \[
        x'=kx.
        \]
        \begin{itemize}
            \item We let $x(t)$ represent the concentration of Plutonium at time $t$.
            \item The rate of change of $x$, $x'(t)$, is related to the current concentration $x$ by a proportion $k$.
        \end{itemize}
        \end{ex}

        We like to choose our variable names in order to best communicate information.  In some cases, $x$ is not the best function name and $t$ is not the best variable name.  Try to understand the role of notation along the way.  One may see equations written differently in specific contexts. Take a look at the next example.

        \begin{ex}{Mechanical Law}{mechanical_law}
        Newton's study of the motion of bodies brought him to say the following.\\

        \emph{The change in velocity of a body is proportional to the force applied divided by the inertial mass of the body.}\\

        The equation that models this is
        \[
        v'(t)=\frac{1}{m} F(t).
        \]
        \begin{itemize}
            \item The $v$ represents the body's velocity at the time $t$ and thus $v'$ is the change in velocity.
            \item The change in velocity should be equal to the applied force, $F(t)$ but also dependent on the objects mass $m$.
            \item We could also describe this equation by noting the fact that $v$ is the derivative of the position $x$. This gives
            \[
            x''(t)=\frac{1}{m}F(t).
            \]
            In other words, this is
            \[
            ma=F.
            \]
        \end{itemize}
        \end{ex}

        \begin{ex}{Harmonic Motion}{harmonic_motion}\index{harmonic motion}
        There are many systems that are not first order.  For example, we might have the following.\\

        \emph{A spring has a rest length $L$. The force on a mass on a spring is proportional to the displacement from this rest length $L$ in the direction opposite the displacement. The force causes an acceleration proportional to the force applied dived by the inertial mass of the body.}\\

        The governing ODE is
        \[
        y''(t) = -\frac{k}{m}(y(t)-L).
        \]
        The variable $y$ was chosen here as this equation can be written in a more standard form with a substitution.
        \end{ex}

        \section{Solutions to an ODE}

        What do we mean when we say that we want to ``solve an ODE?" This means we want to find a function whose derivatives satisfy our ODE.  Let us see a few examples.

        \begin{ex}{Exponential Growth and Decay Solution}{exp_growth_decay_solution}
        Previously, we were given the equation
        \[
        x'(t)=kx(t).
        \]
        I claim that
        \[
        x(t)=Ae^{kt}
        \]
        is a \emph{general solution} to this ODE for any choice of $A$.

        To verify this, we have to take the derivative of our claimed answer $x(t)$. We have,
        \[
        x'(t)=\frac{d}{dt}\left(Ae^{kt}\right)=Ake^{kt}=kx(t),
        \]
        so indeed $x(t)$ is a solution.
        \end{ex}

        \begin{ex}{Mechanical Law Solution}{mechanical_law_solution}
        Let us suppose that there is no force acting on the object.  We should all believe the object should move in a straight line.  With the condition of no force, the equation reads
        \[
        v'(t)=0.
        \]
        I claim that
        \[
        v(t)=c,
        \]
        with $c$ a constant, is a solution to this equation.

        To verify this, take
        \[
        v'(t)=\frac{d}{dt} c = 0.
        \]
        Indeed, this is a solution.  The solution is that of a straight line in space.  We can see this more easily by noting we also have the ODE
        \[
        x'(t) = v(t),
        \]
        since the rate of change of position is velocity. Again, I claim that
        \[
        x(t) = ct
        \]
        is a solution that is a straight line in space.
        \end{ex}

        \begin{exercise}
        Verify that $x(t)$ above is indeed a solution to
        \[
        x'(t)=x(t).
        \]
        Can you see how this means that a particle undergoing no force travels in a straight line?
        \end{exercise}


        \noindent Though we aren't solving equations yet, it is good practice to understand how to see when we have a solution.  Follow my methods in previous examples and do the following exercise.

        \begin{exercise} Earlier, we were given the following
        \[
        y''(t) = \frac{-k}{m} (y(t)-L).
        \]
        Show that
        \[
        y(t) = Ae^{i\sqrt{\frac{k}{m}}t}
        \]
        solves the above differential equation.
        \end{exercise}

        \section{General and Particular Solutions}
        In all previous examples, there were undetermined constants.  These constants appear since, fundamentally, an antiderivative is determined up to a constant.  Though not all ODE are solvable by direct integration, the constants are there due to this reason.

        How do we determine the constants?  It turns out we need a bit more information.  The extra information is also very intuitive and physical.  Take for example, a simplified version of the harmonic oscillator (spring-mass) system
        \[
        u''(t) = -u(t).
        \]
        Notice, we have just removed the constants.  This is always possible to do by picking the right way to measure the problem! Then I stated that the \emph{general solution} \index{general solution} to this ODE is
        \[
        u(t)=Ae^{it}.
        \]
        But, we don't know $A$, which is a complex number. Here, having written the solution this way is possibly confusing.  How is something that's oscillating really being described with complex numbers? It should be said that there are no ``complex measuring sticks" in the real world (that we know of).  So, our answer should end up being real valued in the end! It turns out, this solution is too general for what we think of as being physically real.

        \begin{df}{General Solution}{gen_soln}
            A \boldgreen{general solution} to an $n$th order ODE is a solution with $n$  (call these values $c_1,c_2,\dots,c_n$) undetermined constants. A general solution is in fact a whole family of solutions. That is, there is a solution for each different value of the constants $c_1$, $c_2$, $\dots$, $c_n$.
        \end{df}

        If we think about the situation, we have just determined the general oscillatory behavior of the system, but not the any particular \emph{trajectory} of the system. What we need to know is where we pulled the mass to at the initial time $t=0$. That is, we need
        \[
        x(0)
        \]
        but we also need to know how fast it was moving at that point
        \[
        x'(0).
        \]
        The analogy is as follows: If one is throwing a ball, one needs to know where it is released $\mathbf{x}(0)$, and the velocity at which it is released at $\mathbf{x}(0)$ in order to know where the ball will land.
        \begin{figure}[H]
            \centering
            \includegraphics[width=.7\textwidth]{Figures/projectile-motion.png}
            \caption{Throwing a ball as an example of needing initial data.}
            \label{fig:proj_motion}
        \end{figure}
        \noindent We call these values $x(0)$ and $x'(0)$ the \emph{initial data}.

        \begin{df}{Initial Data}{df: initial_data}
            The \boldgreen{initial data} \index{initial data} to an $n$th order ODE are the specific values for
            \[
            x(0),~ x'(0),\dots,~ x^{(n-1)}(0),
            \]
            where $x^{(k)}$ represents the $k$th derivative of $x(t)$.
        \end{df}

        \noindent It is from this initial data that we can figure out what a \emph{particular solution} to a problem should be.  Up to this point, we have found a general solution which is really a family of solutions. For real problems, we need one answer and not infinitely many.

        \begin{df}{Particular Solution}{particular_solution}\index{particular solution}
            A \boldgreen{particular solution} is one member of the family of general solutions.  That is, a solution where the constants $c_1,~c_2,\dots,~c_n$ are all uniquely determined.
        \end{df}

        \noindent In other words, the particular solution is a single solution to a problem.  Specifically, we call this type of problem where we have initial data present a \boldgreen{initial value problem} \index{initial value problem}.  We will want to make note of this name as we will see \emph{boundary value problems} later on.

        \begin{prop}{Initial Data and Particular Solutions}{initial_data_part_solns}
        In order to find a particular solution to an $n$th order ODE, one needs to know the initial values of the function, its derivative, its second derivative, and all derivatives up to the $(n-1)th$ derivative
        \[
        x(0),~ x'(0),\dots,~ x^{(n-1)}(0).
        \]
        \end{prop}

        \noindent Now, let us work through this a bit.  We can set up an initial value problem and from a general solution we can narrow in on a particular solution.  Try to keep in mind what the initial data is really meaning based on the analogy given before.


        \begin{ex}{Particular Solution to Harmonic Oscillator}{part_soln_harm_osc}
        Consider
        \[
        u''(t)=-u(t)
        \]
        with initial data
        \[
        u(0)=1, ~ u'(0)=0.
        \]
        This has the general solution
        \[
        u(t)=Ae^{it}.
        \]
        We can find the particular solution by rewriting $x$ slightly using Euler's formula. In particular, we find
        \[
        u(t)=a\cos(t)+ib\sin(t)
        \]
        since $A$ could be any complex number.  You could also not make this substitution and find what complex number $A$ has to be.
        \[
        u(0)=a\cos(0)+ib\sin(0)=1
        \]
        from our initial data.  Specifically, this gives us that
        \[
        a=1
        \]
        Then note we also have
        \[
        u'(0)=-a\sin (0) +i b\cos (0)=0
        \]
        which gives us that
        \[
        ib=0 \quad \implies \quad b=0.
        \]
        So our particular solution is
        \[
        u(t)=\cos (t).
        \]
        \end{ex}

        \begin{exercise}
        Plot the solution. Can you determine what is happening if we think of the oscillator as a spring mass system? What does the initial data tell us about the spring and mass at the start?
        \end{exercise}

        \begin{exercise}
            Instead of making the substitution using Euler's formula, solve for the complex number $A$ that shows up in $u(t)=Ae^{it}$.
        \end{exercise}

        The reason why we study these differential systems is to make predictions and models.  Given that, our predictions must be sensible. This means if we are given a differential equation and initial data, there should only be one particular solution.  This is known as \boldgreen{determinism}.     Not all systems are deterministic.  But it turns out the non-deterministic systems are either problematic as models or just very hard to deal with. When solving an ODE, we often call a particular solution a \boldgreen{trajectory}.

        \section{Separable Equations}
        The nicest possible ODEs come from equations that are \emph{separable}\index{separable!differential equation}.  What this means is that, for example, we have a first order equation like
        \[
        x'=f(t)g(x).
        \]
        Why is this nice? Well, in particular, it means that we can simply integrate this equation to solve it.  Many systems exhibit symmetry that allows for this type of separation, so this technique is crucial.

        In general, we have
        \[
        x'=\frac{dx}{dt}
        \]
        and we put
        \begin{align*}
            \frac{dx}{dt}&=f(t)g(x)\\
            \iff \frac{dx}{g(x)} &= f(t)dt.
        \end{align*}
        With this, we can integrate both sides, then solve for $x$. That is, we compute
        \[
        \int \frac{dx}{g(x)} = \int f(t)dt,
        \]
        and we will have an equation where we can isolate $x$

        The nicest separable equations are those which we can integrate directly.  Recall that the ODE
        \[
            x'(t) =f(t)
        \]
        can be solved directly via integration.  That is,
        \[
            x'(t) = \int f(t)dt.
        \]
        Thus, the problem of finding an antiderivative in a first semester calculus course is just one example of a differential equation.  Let's see an example of this.

\begin{ex}{Antiderivative}{antiderivative}
    Consider the equation
    \[
        x'(t) = \sin(t).
    \]
    This equation is separable since the right hand side is only a function of $t$. That is, $g(x)=1$ and $f(t)=\sin(t)$.  Thus, if we follow the separation method, we have
    \begin{align*}
        \frac{dx}{dt} &= \sin(t)\\
        dx &= \sin(t) dt\\
        \int dx &= \int \sin(t) dt\\
        x&= -\cos(t) + c.
    \end{align*}
    This solution we have found is indeed just the antiderivative of $\sin(t)$. Notice that we also pick up an unknown constant of integration. This constant could be determined if we were given an initial condition.
\end{ex}

Antiderivatives are then the easiest possible differential equations since finding their solution amounts to just taking an indefinite integral.  But, separable equations in general will have the same basic technique but with an added $g(x)$ appearing.  The next example will show how you can handle the additional inclusion of the dependent variable $x$ on the right hand side.

        \begin{ex}{Separable ODE}{separable}
        Consider the following ODE where we are trying to find $x(t)$ that solves
        \[
        x'=\frac{t}{x}.
        \]
        Then we can put
        \begin{align*}
            \frac{dx}{dt}&= \frac{t}{x}\\
            \iff xdx &= t dt.
        \end{align*}
        Then we can take the antiderivative of both sides and find
        \begin{align*}
            \int xdx &= \int t dt\\
            \iff \frac{1}{2}x^2 &= \frac{1}{2}t^2 + c\\
            \iff x&=\sqrt{t^2+2c}.
        \end{align*}
        So now we can verify that this is in fact a solution to our ODE.  So we take
        \[
        x'(t) = \frac{d}{dt}\sqrt{t^2+2c}= \frac{t}{\sqrt{t^2+2c}} = \frac{t}{x}.
        \]
        \end{ex}

        \begin{exercise}
        Solve the following ODE using separation
        \[
        x'(t)=t.
        \]
        Note that there will be an undetermined constant that we will learn how to handle next.
        \end{exercise}

        \section{Changing Variables and Symmetry}

        When presented with a differential equation, it can often be in a form that is not the easiest to work with.  Of course, if you do your modelling step correctly you will end up with a meaningful equation.  We want to be able to translate the correct model equation into one that is more workable.

        We found that separable equations were not too bad to solve as we really just need to integrate.  Now, is it possible to turn an equation into a one that is separable? It turns out, we can when the equation satisfies a certain symmetry condition.

        \begin{prop}{Reduction to Separable Equations}{reduction_separable}
            Consider the first order equation
            \[
            x'=f(x,t).
            \]
            If we have that
            \[
            f(x,t)=f(\lambda x, \lambda t)
            \]
            for any number (or function) $\lambda$, then we can reduce the equation to a separable one by defining a new variable $u=\frac{x}{t}$.
        \end{prop}

        The idea of changing variables is immensely important in solving real world problems.  Often times, one can gain insight on the question at hand by searching for these ``better" variables to work in.

        \begin{ex}{Reduction to Separable}{reduction_separable_ex}
            Consider the differential equation
            \[
            x'=\frac{x^2+t^2}{xt}.
            \]
            We can then define
            \[
            f(x,t)=\frac{x^2+t^2}{xt}.
            \]
            Now, we can check that $f(x,t)=f(\lambda x, \lambda t)$ for any $\lambda$ by plugging in.  We have
            \begin{align*}
            f(\lambda x, \lambda t) &= \frac{(\lambda x)^2+(\lambda t)^2}{(\lambda x)(\lambda t)}\\
            &= \frac{\lambda^2(x^2+t^2)}{\lambda^2 xt}\\
            &= \frac{x^2+t^2}{xt}\\
            &=f(x,t).
            \end{align*}
            So our differential equation satisfies the necessary symmetry condition.  So, we can let $u=\frac{x}{t}$ which also means that $x=tu$ and hence
            \[
            f(x,t)=f(tu,t)=\frac{t^2u^2+t^2}{ut^2}=\frac{u^2+1}{u}.
            \]
            Note that given $x=tu$ we have $x'=u+tu'$ which we can now plug into our original expression
            \[
            x'=f(x,t)
            \]
            to get
            \[
            u+tu'=\frac{u^2+1}{u}.
            \]
            Then we can rearrange to isolate $u'$ on the right hand side
            \begin{align*}
                u+tu'&=\frac{u^2+1}{u}\\
                tu'&= \frac{u^2+1}{u}-u\\
                u'&= \frac{1}{tu}.
            \end{align*}
            This is now a separable equation! So, we can solve this in the typical way by noting we have $u'=\frac{du}{dt}$ and putting
            \[
            udu=\frac{dt}{t}.
            \]
            Then we can integrate and solve for $u$
            \begin{align*}
                \int udu &= \int \frac{dt}{t}\\
                \frac{1}{2}u^2 &= \ln(t)+c\\
                u^2&= 2\ln(t)+2c\\
                u&=\pm \sqrt{2\ln(t)+2c}.
            \end{align*}
            Recall that $u=\frac{x}{t}$ and so we have
            \begin{align*}
                \frac{x}{t}&= \pm \sqrt{2\ln(t)+2c}\\
                x&= \pm t\sqrt{2\ln(t)+2c}.
            \end{align*}
            This $x$ is our general solution to the original problem.
        \end{ex}


        Another way that symmetry can help us is more from a modelling perspective. For example, it is possible to make more clever measurements of your system! This is a much needed tool for someone keen on doing experiments. For now, do your best to follow the logic of the following substitution.  This is an example of how one can view a problem in a new way and make it easier.  We should all love symmetry like this!

        \begin{ex}{Changing Variables in Harmonic Oscillator}{changing_var_harmonic}
        Consider the harmonic oscillator equation given by
        \[
        y'' = -\frac{k}{m}(y(t)-L).
        \]
        This equation, being second order, is immediately more difficult to solve.  What we can do, however, is make a change of variables to
        \[
        x(t)=y(t)-L
        \]
        and note that
        \[
        x''(t)=y''(t)
        \]
        but the ODE changes to
        \[
        x''(t)=\frac{k}{m}x(t).
        \]
        This is much easier to solve.  The idea of changing variables is extremely helpful.

        In these new variables, we can see the following figure.
        \begin{figure}[H]
            \centering
            \includegraphics[width=.3\textwidth]{Figures/spring-mass.png}
            \caption{Spring-Mass system in the new variables. The dashed line represents the rest length $L$.}
            \label{fig:spring_mass}
        \end{figure}
        Interestingly enough, the rest length $L$ does not enter the new equation now. As it turns out, it is essentially a useless parameter for understanding the problem.  However, an engineer would care about changing back to our original variable $y$ as the length of a spring factors into design!

        How could you know to make this change? Two ways.  First, one can observe how a spring mass system oscillates.  Set up the experiment and watch for yourself, if you'd like.  What you'll see is that the mass oscillates in a symmetric way about the rest length of the spring. The experimentalist mindset will then suggest that you make the measurement about this position! The other way to observe this is as a mathematician would.  In our original expression you can notice that we have $y(t)-L$ as a quantity and that
        \[
        \frac{d}{dt} (y(t)-L) = y'(t).
        \]
        This may be less intuitive to see, but this tells us that we can safely exchange the quantity $y(t)-L$ for a new quantity $x(t).$
        \end{ex}

        Earlier I claimed that this simplified equation
        \[
        x'' = -\frac{k}{m}x
        \]
        is equivalent to
        \[
        u''=-u
        \]
        by changing the units in which we measure the problem.  Indeed, consider the change of variables
        \[
        u(t)=x\left( \sqrt{k}{m}\right).
        \]
        Then
        \[
        u'(t)=\sqrt{k}{m}x\left(\sqrt{k}{m}\right) \qquad \textrm{and} \qquad u''(t)=\frac{k}{m}x\left( \sqrt{k}{m}\right).
        \]

        \begin{exercise}
        Using the substitution shown above, show that the equation
        \[
        x'' = -\frac{k}{m}x
        \]
        is equivalent to
        \[
        u''=-u.
        \]
        \end{exercise}

        \section{Qualitative analysis for first order equations}

            We have found that we are able to solve a specific type of ordinary differential equation.  Of course, this is a great option when we are able to do so. But, not every equation will be nice or even possible to solve by hand.  Take for example, the equation
            \[
            x'=\sin(tx).
            \]
            This equation is not separable and we will not develop a method for solving this type of equation in the future. However, we can attempt to understand what a solution to this equation may look like even if we do not arrive at a specific function as a solution. This idea falls into the category of \boldgreen{qualitative analysis}\index{qualitative analysis} since we are not getting an exact numerical answer but rather a reasonable guess to a solution that seeks to be qualitatively accurate.


            The first method we will develop here is to plot the \boldgreen{slope field} \index{slope field} of the system.  Specifically, at each point in the $tx$-plane we plot a small segment of a line with a slope given by the value of $x'$ at that point $(t,x)$.  This is perhaps most easily seen with an example with a problem we've already solved.

            \begin{ex}{Slope field for exponential growth}{slope_field_exp_growth}
                Let us consider the exponential growth equation given by
                \[
                x'=x.
                \]
                We know that the general solution to this equation is
                \[
                x=Ae^t,
                \]
                and the particular solution is found when we specify an initial condition. For sake of example, let $x(0)=1$ so that $A=1$ and hence our particular solution is
                \[
                x(t)=e^t.
                \]

                Now, we can plot the slope field by plotting small line segments with slopes equal to the value of $x'$ at each point $(t,x)$ in the $tx$-plane. This gives us
                \begin{figure}[H]
                    \centering
                    \includegraphics[width=.6\textwidth]{Figures_Part_1/slope_field_exp_growth.png}
                    \caption{Slope field is plotted in red with the particular solution $x(t)$ plotted as the black curve.}
                \end{figure}

                The take home idea is that the solution curve will have a slope that is equal to the slope field at each point $(t,x)$. In the above picture, you can see that the curve then follows the slope field that we plotted. To do this yourself, you would pick an initial value which would specify where you start along the vertical axis, and you would draw a curve that follows the slope field.
            \end{ex}

            To practice this yourself, let us return to the earlier example with the equation
            \[
            x'=\sin(tx).
            \]
            If we plot the slope field, we get the following figure.
            \begin{figure}[H]
                \centering
                \includegraphics[width=.6\textwidth]{Figures_Part_1/slope_field_nonlinear.png}
            \end{figure}
            Begin with the initial condition that $x(0)=1$, and attempt to draw your best guess to the solution from this slope field.

            We will revisit this idea later and in more detail when we talk about curves in higher dimensions.  In a more general sense, what we are doing here is allowing our curve to follow a \emph{vector field}.  In other words, these solutions are truly the flow of a vector field.  This is analogous to how a light particle is blown around in a wind -- it flows along the direction of the wind current.



            \subsection{Autonomous equations}

                In physical systems that conserve some quantity in time (typically energy), we will see that the differential equations assume a certain form.  This form is of an \emph{autonomous} differential equation. In fact, this is another type of symmetry that can arise in a differential equation.

                \begin{df}{Autonomous ODE}{autonomous_ode}
                    A first order differential equation is \boldgreen{autonomous}\index{autonomous!differential equation} if it can assume the form
                    \[
                    x'=f(x).
                    \]
                \end{df}
                \noindent A first order autonomous equation is also separable. So there is not much more here to say as far as finding solutions. However, we can also analyze the systems in a qualitative way.  This is beneficial when the right hand side function $f(x)$ is something that we aren't able to integrate analytically.

                When we are given a first order autonomous equation, we can make a plot of the system in the \boldgreen{phase line}\index{line!phase}. That is, we can make a plot of $f(x)$ in the $x'x$-plane.  The reason why we do this is to seek out the following.

                \begin{df}{Equilibria}{equilibria}
                    The \boldgreen{equilibria}\index{equilibria} of an autonomous differential equation is the set of points $x$ where the derivative $x'=0$. That is, these are the points where the system no longer changes over time.
                \end{df}

                Surprisingly, using the phase line will allow us to analyze the behavior of a function without having to arrive at a solution. Again, this is a type of qualitative analysis. We will seek to determine the equilibria of a system since these are points of interest. For systems that are evolving over time, knowing about the equilibria can tell us about the long time behavior of a system.

                Take for example the exponential growth equation:
                \[
                x'=x,
                \]
                and note that if we have the initial condition $x(0)=0$, then the particular solution to this initial value problem is $x(t)=0$.  So, the system stays at the same point for all possible times.  There is no evolution happening!

                Now, how could we have deduced this qualitatively? Well, if we plot the graph of the function $f(x)=x$ in the $x'x$-plane, we notice that $f(x)=0$ when $x=0$.  It is exactly when $f(x)=0$ that we have $x'=0$ and thus there can no longer be motion for the system! Let's take a look at a more complicated example.



                \begin{ex}{Phase line for an autonomous equation}{phase_line_autonomous}
                Let's consider the equation
                \[
                x' = -\left(x-\frac{1}{2}\right)^2+\frac{1}{4}.
                \]
                This equation is indeed autonomous. However, finding a general solution analytically is rather difficult (but it is possible).  Rather, what we can do is plot the graph of $f(x)=-\left(x-\frac{1}{2}\right)^2+\frac{1}{4}$ in the $x'x$-plane as follows.
                \begin{figure}[H]
                    \centering
                    \includegraphics[width=.6\textwidth]{Figures_Part_1/for_phase_line.png}
                \end{figure}
                What we can do next is mark along the $x$ axis where the function is zero as these are the equilibrium of the system. For this system, the set of equilibria is $x=0$ and $x=-1$. Now, how do we know what happens over time? If we pick any value of $x$, we can know the value for $x'$ since we can plug in the value of $x$ into the autonomous equation.  This leads us to the phase line below.

                \begin{centering}
                \begin{tikzpicture}[thick, scale=3]

                    \DrawHorizontalPhaseLine[$x$]{-1,0,1,2}{0.25,0.5,0.75}{-0.25,-0.5,-0.75,1.25,1.5,1.75}
                    \draw [domain=-0.5:1.5,smooth,variable=\x,red] plot (\x,{-1*(\x-0.5)*(\x-0.5)+.25});
                \end{tikzpicture}
                \end{centering}

                Here, I plot $f(x)$ again and we take note of the equilibria at $x=0$ and $x=1$. But, we can also see where $f(x)>0$. For this $f$, the $x$ values in the interval $(0,1)$ lead to $f(x)>0$.  So, we draw arrows pointing in the positive $x$ direction. Similarly, $f(x)<0$ when $x<0$ and $x>1$, so we draw arrows pointing to the negative $x$ direction in that case.

                These arrows tell us which direction our solution will move over time.  For example, if I take $x(0)=\frac{1}{2}$, then as I let $t\to \infty$, $x(t)\to 1$ since the arrows point towards the equilibrium at $x=1$.  If I instead took $x(0)=\frac{3}{2}$ then the arrows will dictate that our solution $x(t)$ approaches the equilibrium $x=1$ as $t\to\infty$ as well.  In this case, we say that the equilibrium $x=1$ is \boldgreen{stable}\index{stable!equilibria}.  The last region of interest could be studied by choosing the initial condition $x(0)=\frac{-1}{2}$.  Notice here that as $t\to \infty$ we have $x(t)\to -\infty$.  This leads us to conclude that the equilibrium $x=0$ is \boldgreen{unstable}\index{unstable!equilibria} since any initial condition around this equilibrium moves \emph{away} from this equilibrium.
                \end{ex}

                The prototypical example to keep in mind is a rigid swinging pendulum.  There, we can note that a pendulum hanging straight down is a stable equilibrium state. Over time, if we start swinging a pendulum, the pendulum will ultimately come to rest at the bottom.  We can remark that this is much like the $x=1$ case in the example system.  Likewise, the pendulum could also rest perfectly in an upright position, but any small movement would cause the pendulum to swing and move away from that equilibrium position.  This means that the top resting position is unstable. See the following figure.

                \begin{figure}[H]
                    \centering
                    \includegraphics[width=.5\textwidth]{Figures_Part_1/pendulum.png}
                    \caption{The two equilibria for a rigid pendulum.}
                \end{figure}

                This type of qualitative analysis is important.  To get a feel for it yourself, you may want to consider analyzing the system
                \[
                x' = \sin(x).
                \]
                \begin{exercise} Do the following exercises for the above system $x'=\sin(x)$.
                \begin{enumerate}[(a)]
                    \item Draw the phase line for this system.
                    \item Find all (infinitely many) equilibria.
                    \item Classify which equilibria are stable and which are unstable.
                \end{enumerate}
                \end{exercise}


        \section{First Order Linear Differential Equations}

        Another nice set of equations that we can solve come in the form of linear equations.  In particular, we first consider \emph{first order linear ODEs}\index{linear differential equation} but will later consider the second order version as well.  Generally, linear problems tend to be solvable whereas nonlinear problems are very hard.  One can also learn how to approximate nonlinear problems via linearization, but we do not do that here.

        \begin{df}{First Order Linear ODE}{first_order_lin}
            A \boldgreen{first order linear ODE} is an equation that can assume the form
            \[
            x'+f(t)x=g(t).
            \]
        \end{df}

        There is a general technique for solving this type of equation and, in fact, a general technique for solving higher order linear equations.  More on that later.  For now, let us work to solve this problem.  The method we will use utilizes a function called an \boldgreen{integrating factor}\index{integrating factor}. The idea is that we can use the format of the equation to our advantage.

        \subsubsection{Integrating Factor}
        Consider a first order linear ODE given by
        \[
        x'+f(t)x=g(t).
        \]
        Then, multiply the whole expression by a yet undetermined function $\mu(t)$ to get
        \begin{equation}
        \mu(t)x'+\mu(t)f(t)x=\mu(t)g(t). \label{eq:int_fact_1}
        \end{equation}
        The reason we have done this is that we can now take a look at the derivative of the product
        \begin{equation}
        \left( \mu(t)x\right)'= \mu'(t)x+\mu x'. \label{eq:int_fact_2}
        \end{equation}

        From here, we can set this product derivative (right hand side of \ref{eq:int_fact_2}) equals the left hand side of expression \ref{eq:int_fact_1}.  This gives us
        \[
        \mu'(t)x+\mu x' = \mu(t)x'+\mu(t) f(t) x
        \]
        which means that
        \[
        \mu'(t)x=\mu(t) f(t) x.
        \]
        This is a separable ODE! So we can solve this for $\mu$ using the separation technique. This $\mu$ is the integrating factor.

        \begin{exercise}
            Solve the separable ODE
            \[
            \mu'(t)x=\mu(t)f(t)x
            \]
            for $\mu$ and show that you find
            \[
            \mu(t) = e^{\int f(t)dt}.
            \]
        \end{exercise}

        \noindent From the exercise, we have that
        \[
        \mu(t)=e^{\int f(t)dt}
        \]
        and we can use this to complete the problem.  Specifically, we have that the left hand side of \ref{eq:int_fact_2} is equal to the right hand side of \ref{eq:int_fact_1} to get
        \[
        (\mu(t) x)' = \mu(t)g(t).
        \]
        We can integrate both sides and solve for $x$ to find
        \[
        \boxed{x = \frac{1}{\mu(t)}\int \mu(t) g(t)dt.}
        \]
        The above expression for $x$ tells us the general solution to any first order linear ODE.

        \begin{ex}{Solving an ODE with Integrating Factor}{solving_ode_int_fact}
            Consider the first order equation
            \[
            x'+\frac{2x}{t}=2\cos(t).
            \]
            Note that we can say $f(t)=\frac{2}{t}$ and then
            \begin{align*}
            \mu(t)&=e^{\int \frac{2}{t}dt}\\
            &= e^{2\ln(t)}\\
            &=t^2.
            \end{align*}
            Note, when computing the integrating factor $\mu$ we do not need to have a $+c$ after integrating. It will cancel later on if you do include it. Next, we note that
            \[
            x=\frac{1}{\mu(t)} \int \mu(t) g(t)dt
            \]
            where $g(t)=2\cos(t)$ in this case.  Thus
            \begin{align*}
                x&=\frac{1}{t^2} \int t^2 \cdot 2\cos(t)dt\\
                &=\frac{2}{t^2} \left(t^2\sin(t)+2t\cos(t)-2\sin(t)+c\right),
            \end{align*}
            where the last equality involved using integration by parts twice.  So we have found a general solution to our original ODE.
        \end{ex}

        \begin{exercise}
            Complete the integration by parts in the above example.
        \end{exercise}

        Just to introduce some terminology, we can note that if we have a first order linear ODE of the form
        \[
        x'+f(t)x=0
        \]
        we call this equation \boldgreen{homogeneous}\index{homogeneous!differential equation}.  Otherwise, we have that the expression with a nonzero right hand side
        \[
        x'+ f(t)x=g(t)
        \]
        is called \boldgreen{inhomogenous}\index{inhomogeneous!differential equation}.  The homogeneous case for first order equations are simply separable equations and the distinction here is not really necessary.

        \section{Applications to Chemical Kinetics}
        \index{chemical kinetics}
        As a chemist, one would probably like to understand how we can model elementary chemical reactions with mathematics.  For example, maybe we would like to understand the rate of reaction of hydrogen $H_2$ and oxygen $O_2$ to create water $H_2 O$. We typically write
        \[
        2H_2 + O_2 \to 2H_2O.
        \]
        \noindent Of course, we should actually be writing
        \[
        2H_2 + O_2 \leftrightarrows 2H_2O
        \]
        since real reactions have an equilibrium.  The amount of back and forth in a reaction also depends on parameters like temperature, pressure, or concentration.

        In thinking about these equations, one should consider how the rate of change of the reactants relates to the rate of change of the products.  Likewise, the coefficients before a molecule describe how number of molecules that may be required to create a reaction or are produced from that reaction.  For example, the creation of water requires two hydrogen molucules ($2H_2$) and just one oxygen molecule ($1O_2$).  This means that we should expect to be using up twice as many molecules of hydrogen per second when we compare it to the rate of change of oxygen.  Likewise, for every reaction that takes place, we produce two water molecules ($2H_2O$), and as such, we expect that the number of water molecules will be created at the same rate that we are using up hydrogen.

        \begin{question}
            Write down another chemical reaction of your choice.  Before continuing on with this section, think about how the rates of the products and reactants will be related to each other. Can you see why we are studying this with differential equations?
        \end{question}

        For now, let us assume we are looking at the original model which can we written as a sum of $m$ reactants $R_i$ each with an amount $r_i$ that give us an amount $p_i$ of $n$ different products $P_j$. That is, we have a reaction
        \begin{align*}
            r_1R_1 + r_2R_2 + \cdots + r_mR_m \to p_1 P_1 + p_2P_2 +\cdots p_n P_n,
        \end{align*}
        that gives us an equation
        \[
        \sum_{i=1}^m r_iR_i = \sum_{j=1}^n p_jP_j.
        \]
        The second is just a mathematically succinct way of representing the quantitites in the reaction. We will call the $r_i$ and $p_j$ the \boldgreen{stoichiometric variables}.  These amounts $r_i$ represent how many of the $i^\textrm{th}$ reactants we need to create a single reaction and the $p_j$ represent how many of the $j^\textrm{th}$ products are produced given a single reaction has occured.

        We often write the equation above in the following form:
        \begin{equation}
        0=\sum_{j=1}^n p_j P_j - \sum_{i=1}^m r_iR_i. \label{eq:stoich}
        \end{equation}
        Later, we refer to this equation and it has been numbered so that we keep track of it as we derive an expression for the rate of change for each of the reactants and products.

        Suppose we start with an initial amount of substance $A$, $N_A(0)$. Then we say that the number of molecules of species $A$ at time $t$ is $N_{A}(t)$. Note that $N_A$ may represent either a reactant or a product and we let $a$ be the amount of species $A$ necessary for the reaction to take place.  That is, if $A$ is a reactant, $a$ is the number of molecules of species $A$ required to form a reaction and if $A$ is a product then $a$ is the number of $A$ produced by the reaction. In the case that $A$ is a reactant, we will have $a<0$ and if $A$ is a product then $a>0$. The amount of reaction observed is given by the \boldgreen{extent of reaction} $\xi$ defined by
        \[
        N_A(t) = N_A(0)+a\xi.
        \]
        Here, $\xi$ is some positive number that represents how much reaction has taken place.  We will be using this variable later to relate different species involved in the reaction.

        In \ref{eq:stoich}, we can see that the total number of species has to be conserved. That is, our reaction is never losing any of the individual atoms in any way.  Using this assumption, we have that the \boldgreen{rate of conversion} of reactants to products is
        \begin{align*}
            \rho &= \frac{d\xi}{dt}\\
            &= \frac{1}{a}\frac{dN_a}{dt}.
        \end{align*}
        In other words, if we know how fast one species is changing, then we know how fast the others must be changing as well.  Remember, the reaction is very strict.  We always require the correct number of reactants to produce the given set of products.  These rates must be related to one another.

        In the case that we care about concentrations of chemicals as opposed to the amount of molecules we let
        \[
        [A]=\frac{N_A}{V}
        \]
        where $V$ is the volume the substance $A$ is contained in.  Then we have
        \[
        v=\frac{\rho}{V}=\frac{1}{a}\frac{d[A]}{dt}.
        \]
        Now, for a general reaction given by
        \[
        r_1R_1 + r_2R_2 + \cdots + r_mR_m \to p_1 P_1 + p_2P_2 +\cdots p_n P_n,
        \]
        we must have that the concentrations of each species must change with matching rates.  We can write this mathematically by
        \[
        v=-\frac{1}{r_i}\frac{d[R_i]}{dt}=\frac{1}{p_j}\frac{d[P_j]}{dt},
        \]
        for every product $P_j$ and reactant $R_i$. Note that since reactants are utilized to create products, their rate of conversion is negative since their concentration must decrease.  Similarly, products have a positive rate of conversion since they are being created by the reaction process.

        \begin{ex}{$H_2$ and $O_2$ to $H_2O$}{water}
        If we are wishing to model
        \[
        2H_2 + O_2 \to 2H_2O
        \]
        we will have the equations
        \begin{align*}
            v=-\frac{1}{2}\frac{d[H_2]}{dt}=-\frac{d[O_2]}{dt}=\frac{1}{2}\frac{d[H_2O]}{dt}.
        \end{align*}
        \end{ex}

        At this point, we do not have enough data to create an equation that makes predictions for us. Indeed, we have the related rate expressions for each species, but we do not know what the rate of conversion $v$ itself is. However, we can determine $v$ from experimentation in a lab.  There, we find that
        \[
        v=k[R_1]^{r_1}[R_2]^{r_2}\cdots
        \]
        which allows us to complete our model. This expression here should make some sense.  If we have the reaction
        \[
            2H_2 + O_2 \to 2H_2 O,
        \]
        then we would have $v=k[H_2]^2 [O_2]$ which is describing the likelihood of two hydrogen molecules coming into close contact with a single oxygen molecule.  In this expression, we call $k$ the \boldgreen{rate constant} of the reaction and note that we have $k>0$. The numbers $r_i$ determine that \boldgreen{order of the reaction}. We say that $r_i$ is the order with respect to the reactant $R_i$. For example, we may have
        \begin{itemize}
            \item First order: $R \to \textrm{Products}$;
            \item Second order: $2R \to \textrm{Products}$;
            \item Second order: $R_1+R_2 \to \textrm{Products}$.
        \end{itemize}
        Thus, one can realize the order of the reaction by the expression
        \[
            \textrm{Order} = \sum_{i=1}^m r_i.
        \]
        That is, the order of the reaction is the total number of molecules of reactants required to create a single reaction.

        \begin{ex}{Creation of water}{creation_of_water}
            Consider the chemical reaction
            \[
                2H_2 + O_2 \to 2H_2 O.
            \]
            This is a third order reaction since we require three reactant molecules to create products. The reaction rate is then
            \[
                v=k[H_2]^2 [O_2].
            \]
            To find equations for rate of change the concentrations, we use the other expression for $v$ found in Example \ref{ex:water}. Thus we have the equations
            \begin{align*}
                \frac{d[H_2]}{dt} &= -2 k [H_2]^2 [O_2]\\
                \frac{d[O_2]}{dt} &= -k[H_2]^2[O_2]\\
                \frac{d[H_2O]}{dt} &= +2k [H_2]^2[O_2].
            \end{align*}
            In the expression above, we would need to find an numerical value for the rate constant $k$. Once we have that, we then need to handle the fact that the concentrations are \emph{coupled}. That is, for example, the rate of change of $[H_2]$ depends on the current amount of $[O_2]$.  We will revisit this concept later. Lastly, we would need to know the initial concentrations of each species and we could determine a particular solution from there!
        \end{ex}

        This formulation of chemical kinetics is powerful.  It will give us a method to write down the rate of reaction for even the most complicated of chemical reactions.  Whether we can solve these by hand is a different question, but nonetheless we can create a differential equation just from the structure of the reaction! The important constant in the expression is the rate constant $k$.  This is a quantity that a chemist would measure in a lab by seeing how long a reaction takes to complete.  Once this is determined for multiple reactions, we can then use this information to investigate more complicated kinetics.

        The rest of this section is devoted to working through some example problems.  Along the way, you may want to find reactions that fit the description provided.  Look up the rate constant for those reactions and see if you can then take the general problems and apply them to your specific reaction. If done properly, this would be able to tell you how quickly you would expect the reaction to happen.

        \begin{ex}{First Order Reaction}{first_order_react}
        Consider a reaction
        \[
        R \to \textrm{Products}
        \]
        with an initial concentration of $R$ given by $[R]_0$. We then have
        \[
        v=\frac{-d[R]}{dt}=k[R].
        \]
        For ease of notation, let $x=[R]$ and we have
        \[
        x'=-kx,
        \]
        which we have solved before.
        \end{ex}

        \begin{exercise}
        Either solve the above differential equation again, or find the solution to the initial value problem somewhere in this text.
        \end{exercise}

        \begin{ex}{Second Order Reaction}{second_order_react}
        Consider a reaction
        \[
        2R \to \textrm{Products}
        \]
        with an initial concentration $[R]_0$.  Then we have
        \[
        v=-\frac{1}{2} \frac{d[R]}{dt}=k[R]^2.
        \]
        Again, let $x=[R]$ and we have the ODE
        \[
        x'=-2kx^2.
        \]
        \end{ex}

        \begin{exercise}
        Find the particular solution to the initial value problem posed in the previous example.
        \end{exercise}


        %% THERE MAY BE MISTAKES HERE
        \begin{ex}{Several Reactions}{sev_react}
        Consider a chain of reactions as follows
        \[
        A\xrightarrow{k_1} B \xrightarrow{k_2} C.
        \]
        Then we wish to describe the concentrations of each species $A$, $B$, and $C$ over time. We know that we have
        \[
        \frac{d[A]}{dt}=-k_1[A]
        \]
        and it follows that
        \[
        \frac{d[B]}{dt}=k_1[A]-k_2[B].
        \]
        Then, at time $t=0$ we have initial concentrations $[A]_0$, $[B]_0=0$, and $[C]_0$ so that we are only starting with species $A$. Later on, at time $t$, we have $[A]=[A]-x$ and $[B]=y$ which means that we have $[C]=x-y$.  Note that $[C]$ is created by $A\to B$ and from $B\to C$ so we must subtract off the concentration of $B$ that has not converted to species $C$ yet. This gives us the equations
        \begin{equation}
                        \frac{d([A]_0-x)}{dt}=-k_1([A]-x) \label{eq:chem_a}
        \end{equation}
        \begin{equation}
                        \frac{dy}{dt}=k_1([A]_0-x)-k_2y. \label{eq:chem_b}
        \end{equation}
        Note that \ref{eq:chem_a} is a separable equation with solution
        \[
        [A]_0-x=[A]_0e^{-k_1t}.
        \]
        We can then substitute in this solution to \ref{eq:chem_b} to get
        \[
        \frac{dy}{dt}=[A]_0k_1e^{-k_1t}-k_2y
        \]
        which gives us the first order linear equation
        \[
        y'+k_2y=[A]_0k_1e^{-k_1t}.
        \]
        To solve this, we find the integrating factor
        \[
        \mu(t)=e^{\int k_2 dt}=e^{k_2t}.
        \]
        Then we have that
        \[
        y=\frac{1}{e^{k_2t}}\int e^{k_2t}[A]_0k_1e^{-k_1t}dt=[A]_0k_1e^{-k_2t}\int e^{(k_2-k_1)t}dt.
        \]
        This integral comes down to two cases; first when $k_1=k_2$ and when $k_1\neq k_2$. Thus we have
        \[
        y=\begin{cases}
        \frac{[A]_0k_1}{k_2-k_1}e^{-k_1t}+ce^{-k_2t} & k_1\neq k_2\\
        [A]_0k_1te^{-k_1t}+ce^{-k_2t} & k_1=k_2.
        \end{cases}
        \]
        Using our initial conditions, we have $y=0$ at time $t=0$ which gives us the particular solution
        \[
        y=\begin{cases}
        \frac{[A]_0k_1}{k_2-k_1}\left( e^{-k_1t}-e^{-k_2t}\right) & k_1\neq k_2\\
        [A]_0k_1te^{-k_2t} & k_1=k_2.
        \end{cases}
        \]

        Now, we will have that each concentration can be written as
        \begin{align*}
            [A]&=[A]_0e^{-k_1t}\\
            [B]&=\begin{cases}
        \frac{[A]_0k_1}{k_2-k_1}\left( e^{-k_1t}-e^{-k_2t}\right) & k_1\neq k_2\\
        [A]_0k_1te^{-k_2t} & k_1=k_2.
        \end{cases}\\
        [C]&=[A]_0-[A]-[B].
        \end{align*}
        Now, we can plot the concentrations below letting $[A]$ be in purple, $[B]$ in red, and $[C]$ in black.
        \begin{figure}[H]
    \centering
    \begin{subfigure}[h]{0.3\textwidth}
        \includegraphics[width=\textwidth]{Figures_Part_2/k1=k2.png}
        \caption{$k_1=k_2=1$.}
    \end{subfigure}
    ~
    \begin{subfigure}[h]{0.3\textwidth}
        \includegraphics[width=\textwidth]{Figures_Part_2/k1=1k2=10.png}
        \caption{$k_1=1$ and $k_2=10$.}
    \end{subfigure}
    ~
    \begin{subfigure}[h]{0.3\textwidth}
        \includegraphics[width=\textwidth]{Figures_Part_2/k1=10k2=1.png}
        \caption{$k_1=10$ and $k_2=1$.}
    \end{subfigure}

        \end{figure}

        If instead, we wished to have arbitrary initial concentrations, we would let $[B]_0$ and $[C]_0$ be the initial concentrations for $[B]$ and $[C]$ respectively.  This would give us the equations
        \begin{align*}
            [A]&=[A]_0e^{-k_1t}\\
            [B]&=\begin{cases}
                \frac{[A]_0k_1}{k_2-k_1}e^{-k_1t}+\left( -\frac{[A]_0k_1}{k_2-k_1}+[B]_0\right)e^{-k_2t} & k_1\neq k_2\\
                [A]_0k_1te^{-k_1t}+[B]_0e^{-k_2t} & k_1=k_2.
            \end{cases}\\
            [C]&=([A]_0-[A])+([B]_0-[B])+[C]_0.
        \end{align*}
        You should verify these that these new particular solutions are correct.
        \end{ex}

        \begin{exercise}
        Go through the previous example and fill in the missing steps.
        \end{exercise}

        \section{Second Order Equations}

        Since Newtonian physics is governed by the equation
        \[
        F=ma=mx''
        \]
        we often see differential equations that are second order.  The question is then, how can we solve these?  Also, recall that in general, a second order equation can be written of the form
        \[
            x'' = f(t,x,x').
        \]
        We will immediately seek out special types of second order equations for which we can solve as we noticed we had to do this for the first order equations.

        There are a two main forms of second order equations we will consider, but there are more out there. Let's define the class of equations we will stick with.

        \begin{df}{Second Order Linear ODEs}{second_order_lin}
            A second order ODE is \boldgreen{linear}\index{linear!differential equation} if it can assume the form
            \[
            x''+f(t)x'+g(t)x=h(t).
            \]
            We say that the equation is \boldgreen{homogeneous}\index{homogeneous!differential equation} if $h(t)=0$ and otherwise it is \boldgreen{inhomogenous}\index{inhomogeneous!differential equation}.
        \end{df}

        \begin{remark}
        An inhomogenous second order linear ODE \underline{can} be always be solved with the functions $f$, $g$, and $h$ are smooth enough.  We will not cover solving this general of a problem here.
        \end{remark}

        \subsection{Homogeneous and Constant Coefficients}
        Specifically, we care about homogeneous second order linear ODEs where the functions $f(t)=b$ and $g(t)=c$ are constant.  These will be the most simple to solve yet fairly applicable. An equation like this can be rearranged to take the form
        \[
        x''+bx'+cx=0.
        \]
        A reasonable guess (or \boldgreen{ansatz}\index{ansatz}) for a solution is an exponential function of the form
        \[
        x(t)=Ae^{\lambda t}
        \]
        where $A$ is an arbitrary constant. Since we can see that each successive derivative seems to only multiply our function by a constant, this is a good first guess. If we try to make this function $x$ work to solve our ODE, we plug it in and see that we get
        \begin{align*}
            \left(Ae^{\lambda t}\right)''+b\left(Ae^{\lambda t}\right)'+cAe^{\lambda t}&=0\\
            \iff\lambda^2 Ae^{\lambda t}+\lambda b Ae^{\lambda t} +c Ae^{\lambda t}&=0\\
            \iff Ae^{\lambda t}\left( \lambda^2 + b\lambda +c\right)&=0\\
            \iff \lambda^2 +b\lambda +c &=0.
        \end{align*}
        The last equality must be true since $Ae^{\lambda t}$ is never equal to zero.  Thus, it seems we found a solution to the equation via the roots of this polynomial
        \[
        \lambda^2+b\lambda +c,
        \]
        which we will refer to as the \boldgreen{characteristic polynomial}\index{characteristic polynomial}. The roots of the characteristic polynomial are then able to be found with the quadratic formula to yield
        \[
        \lambda=\frac{-b\pm \sqrt{b^2-4c}}{2}.
        \]
        This leads us to the following.

        \begin{prop}{Solutions to Linear Constant Coefficient Second Order ODE}{solutions_hom_constant}
        Consider the differential equation
        \[
        x''+bx'+cx=0.
        \]
        Then the corresponding characteristic polynomial is
        \[
        \lambda^2+b\lambda +c
        \]
        and we let the solutions of this equation be equal to $\lambda_1$ and $\lambda_2$. Then,
        \begin{itemize}
                \item If $\lambda_1 \neq \lambda_2$, we have the general solution
        \[
        x(t)=C_1 e^{\lambda_1t}+C_2e^{\lambda_2t},
        \]
        where $C_1,C_2,\lambda_1,\lambda_2 \in \C$.
            \item If $\lambda_1=\lambda_2$, we have the general solution
        \[
            x(t) = C_1 e^{\lambda_1 t} + C_2 t e^{\lambda_1 t},
        \]
        where $C_1, C_2, \lambda_1 \in \C$.
        \end{itemize}
        \end{prop}

        \begin{question}
        Above we see that we have a sum of solutions and our ansatz only chose one. Is it always possible to have a sum of different solutions be a solution?
        \end{question}

        \begin{answer}
        Yes. We will see by the following theorem that this is the case! This is an important result for studying quantum systems.
        \end{answer}

        \begin{thm}{Superposition of Solutions}{superposition_solns}
        Let $x_1$ and $x_2$ be general solutions to the equation
        \[
        x''+f(t)x'+g(t)x=0.
        \]
        Then any \boldgreen{linear combination}\index{linear combination} (or \boldgreen{superposition}\index{superposition}) of solutions
        \[
        \alpha_1 x_1 + \alpha_2 x_2
        \]
        with $\alpha_1,\alpha_2\in \C$, is also a solution.\\

        \begin{proof}
        Let $x_1$ and $x_2$ be general solutions to the above ODE. Then consider the function
        \[
        x=\alpha_1 x_1 + \alpha_2 x_2.
        \]
        Then we have
        \begin{align*}
            x''+f(t)x'+g(t)x&= (\alpha_1 x_1 + \alpha_2 x_2)''+f(t)(\alpha_1x_1+\alpha_2 x_2)'+g(t)(\alpha_1x_1+\alpha_2x_2)\\
            &= \alpha_1 x_1'' + \alpha_2 x_2'' + \alpha_1 f(t)x'+\alpha_2 f(t)x_2'+\alpha_1 g(t)x_1 + \alpha_2 g(t) x_1\\
            &= \alpha_1 ( x_1''+f(t)x_1'+g(t)x_1)+\alpha_2(x_2''+f(t)x_2'+g(t)x_2)\\
            &=0, \quad\textrm{since $x_1$ and $x_2$ are solutions.}
        \end{align*}
        Hence $x=\alpha_1x_1 + \alpha_2 x_2$ is also a solution.
        \end{proof}
        \end{thm}

        \begin{remark}
        You may have heard of superposition in quantum mechanics. It turns out that this is exactly what is meant in the mathematical theory. Eventually, we'll see an example of what this physically means in a quantum system.
        \end{remark}

        \begin{ex}{Solving the Harmonic Oscillator}{solve_harmonic}
        Consider the equation
        \[
        mx''=-kx
        \]
        with the initial data $x(0)=1$ and $x'(0)=0$. We have shown that we have a solution to this equation before, but now we can explicitly solve it.  We can rewrite the ODE as a second order linear homogeneous equation with constant coefficients by putting
        \[
        x''+\frac{k}{m}x=0.
        \]
        Then, for sake of notation, let $\omega = \sqrt{\frac{k}{m}}$ so that
        \[
        x''+\omega^2x=0.
        \]
        Then the characteristic polynmomial is
        \[
        \lambda^2+\omega^2
        \]
        and the roots are
        \begin{align*}
            \lambda^2+\omega^2&=0\\
            \lambda^2&=-\omega^2\\
            \iff \lambda&= \pm \sqrt{-\omega^2}= \pm i \omega.
        \end{align*}
        Thus, the general solution is then
        \[
        x=C_1 e^{i\omega t}+C_2e^{-i\omega t}
        \]
        where $C_1,C_2\in \C$.

        Now, we can find the particular solution from the initial data $x(0)=1$ and $x'(0)=0$ and letting
        \[
        C_1=a_1+b_1i \qquad \textrm{and} \qquad C_2=a_2+b_2i.
        \]
        We then have
        \begin{align*}
                    1=x(0)&=(a_1+b_1i)e^{i\omega \cdot 0}+(a_2+b_2 i)e^{-i\omega \cdot 0}\\
                    &=(a_1+b_1i)+(a_2+b_2i)\\
                    &=(a_1+a_2)+i(b_1+b_2).
        \end{align*}
        Thus we must have
        \begin{align*}
            a_1+a_2&=1\\
            b_1+b_2&=0 ~\implies~ b_1=-b_2.
        \end{align*}
        From the other initial condition, we have
        \begin{align*}
        0=x'(0)&=\omega(a_1+b_1i)e^{i\omega \cdot 0}-\omega (a_2+b_2i)e^{i\omega \cdot 0}\\
        &=(a_1-a_2)+i(b_1-b_2).
        \end{align*}
        Thus we have
        \begin{align*}
            a_1-a_2&=0 ~\implies~ a_1=a_2\\
            b_1-b_2&=0 ~\implies~ b_1=b_2.
        \end{align*}
        Now we have $b_1=-b_2$ and $b_1=b_2$ which means $b_1=b_2=0$.  Then, we also have $a_1=a_2$ which we can substitute into $a_1+a_2=1$ to find that $a_1=a_2=1/2$.  Hence we have our particular solution
        \begin{align*}
            x(t)&=\frac{1}{2}e^{i\omega t}+\frac{1}{2}e^{-i\omega t}.
            \end{align*}
            This, we can rewrite to find a more familiar form of the solution by
            \begin{align*}
            x(t)&= \frac{1}{2}(\cos(\omega t)+i\sin(\omega t)) + \frac{1}{2}(\cos(-\omega t)+i\sin(-\omega t))\\
            &= \frac{1}{2}(\cos(\omega t)+i\sin(\omega t))+\frac{1}{2}(\cos(\omega t)-i\sin(\omega t))
            &=\cos(\omega t).
        \end{align*}
        Note that I used the fact that sine is an odd function meaning that $\sin(-x)=-\sin(x)$ and that cosine is an even function meaning $\cos(-x)=\cos(x)$.
        \end{ex}

        \newpage

        \begin{exercise}
        Find where we claimed a solution to the harmonic oscillator earlier in this text and check that this solution matches that.
        \end{exercise}

        \subsection{Qualitative Analysis}

        Solutions to homogeneous second order linear ODEs with constant coefficients only come in a few different classes of solutions. Essentially, they oscillate, exponentially grow or decay, or a combination of the two.  There are some edge cases to be careful of, but they are not typical.  Let's see why this is the case.

        Recall that we have, in general, an equation
        \[
        x''+bx'+cx=0
        \]
        which gives rise to the characteristic polynomial
        \[
        \lambda^2+b\lambda + c.
        \]
        The roots are then
        \[
        \lambda = \frac{-b\pm \sqrt{b^2-4c}}{2}= \frac{-b}{2}\pm \frac{\sqrt{b^2-4c}}{2}.
        \]
        These roots can be real, complex, or purely imaginary. The real part can be greater than zero, or less.  These facts encompass all the solutions we care about.

        \begin{itemize}
            \item \textbf{Case 1:} Consider the case where the roots of the characteristic polynomial are both real and denote the roots by $\lambda_1$ and $\lambda_2$.  If this is the case, then we must have that $b^2>4c$ so that the square root in the quadratic formula is not of a negative number. For this, we have two subcases.
            \begin{itemize}
                \item \textbf{Subcase 1:} If we have two distinct real roots, $\lambda_1$ and $\lambda_2$ where $\lambda_1\neq \lambda_2$, then our general solution to the equation is
                \[
                x(t)=C_1e^{\lambda_1 t}+C_2e^{\lambda_2 t}.
                \]
                Note that $\lambda_1$ and $\lambda_2$ could be both positive, both negative, one negative one positive, or either could be zero as well.

                Let's say we have $\lambda_1=1$, $\lambda_2=-1$ and let $C_1=C_2=1$. Then the solution plotted in the $xt$-plane looks like
                \begin{figure}[H]
                    \centering
                    \includegraphics[width=.7\textwidth]{Figures_Part_2/l1=1_l2=-1.png}
                \end{figure}
                \item \textbf{Subcase 2:} If we have two identical real roots, $\lambda=\lambda_1=\lambda_2$ then the general solution is
                \[
                x(t)=C_1 e^{\lambda t}+C_2te^{\lambda t}.
                \]

                Here, we can take $\lambda=1$ and let $C_1=C_2=1$ and plot the solution:
                \begin{figure}[H]
                    \centering
                    \includegraphics[width=.7\textwidth]{Figures_Part_2/l=1.png}
                \end{figure}
                We could also take $\lambda=-1$ with $C_1=C_2=1$ to see:
                \begin{figure}[H]
                    \centering
                    \includegraphics[width=.7\textwidth]{Figures_Part_2/l=-1.png}
                \end{figure}
            \end{itemize}
            \item \textbf{Case 2:} Consider the case where the roots to the characteristic polynomial are complex valued.  That happens when we have $b^2-4c$ since then we will have a square root of a negative number appear with the quadratic formula. In this case, if we have a root $\lambda$, then $\lambda^*$ is \underline{always} the other root (take a look at the quadratic formula, and see if you can see why this is the case).  Thus, we can put $\lambda=\alpha+\beta i$ and then have another root $\lambda^*=\alpha-\beta i$. This gives us the general solution
            \begin{align*}
                x(t)&= C_1 e^{\lambda t}+C_2 e^{\lambda^* t}\\
                &= C_1 e^{\alpha t}e^{i\beta t}+C_2e^{\alpha t}e^{-i\beta t}\\
                &= e^{\alpha t}\left( C_1 e^{i\beta t}+C_2 e^{-i\beta t}\right).
            \end{align*}
            Here, we must have that $C_1$ and $C_2$ are complex numbers.  We can also rewrite this general solution as
            \[
            x(t)= e^{\alpha t}(C_1 \cos(\beta t)+C_2\sin(\beta t)).
            \]

            Here we can take $\alpha = 1$ and $\beta =5$ with $C_1=C_2=1$ and plot the solution in the $xt$-plane to see:
            \begin{figure}[H]
                \centering
                \includegraphics[width=.7\textwidth]{Figures_Part_2/a=1-b=5.png}
            \end{figure}
            We could also take the case with $\alpha=0$, $\beta=5$, and $C_1=C_2=1$ and plot:
            \begin{figure}[H]
                \centering
                \includegraphics[width=.7\textwidth]{Figures_Part_2/a=0-b=5.png}
            \end{figure}
            Lastly, we could plot with $\alpha=-1$, $\beta =5$ and $C_1=C_2=1$ to see:
                        \begin{figure}[H]
                \centering
                \includegraphics[width=.7\textwidth]{Figures_Part_2/a=-1-b=5.png}
            \end{figure}
        \end{itemize}

        \begin{exercise}
        Work out why if the roots to the characteristic polynomial are complex, that we always have the roots $\lambda$ and $\lambda^*$.
        \end{exercise}

        \begin{exercise}
        Show that the two general solutions
        \[
        x(t)=e^{\alpha t}(C_1 e^{i\beta t}+C_2e^{-i\beta t})
        \]
        and
        \[
        x(t)=e^{\alpha t}(C_1 \cos(\beta t)+C_2\sin(\beta t))
        \]
        are equivalent using Euler's formula.
        \end{exercise}

        \subsection{Inhomogeneous Linear Equations}

        More often than not, we look at physical systems where there is some external force or potential that causes the system to change over time.  In the case of second order linear equations, this corresponds to the inhomogeneous equation
        \[
        x''+bx'+cx=F(t)
        \]
        where we think of $F(t)$ as an external force.  For specific forcing terms $F(t)$, we can solve this equation exactly using the \boldgreen{method of undetermined coefficients}\index{method of undetermined coefficients}.  The idea is that we can solve the homogeneous equation
        \[
        x_h''+bx_h'+cx_h=0
        \]
        to find a solution $x_h(t)$.  This homogeneous solution can be understood as describing the dynamics of the system when undergoing no external forces. For example, if I have a mass on a spring, the system will oscillate even if I supply no external forces to it. We refer to this as the \boldgreen{homogeneous solution}\index{homogeneous solution}. It does not, however, solve the equation on its own. We then need another function which we call the \boldgreen{particular integral}\index{particular integral} and denote by $x_p(t)$. This function is given by an ansatz based on what the forcing term is.  This particular integral is essentially modeling the response that a system will have when it interacts with an external force.  In this case, $x_p(t)$ solves the inhomogeneous equation
        \[
        x_p''+bx_p'+cx_p=F(t)
        \]
        but is needs to be accompanied by a homogeneous solution. The solution to an inhomogeneous equation as above will then be
        \[
        x=x_h+x_p.
        \]
        This means that the dynamics for a system come down to a combination (or superposition) two different parts.  First, the system itself has its own behavior whether acted on by a force or not.  This is the $x_h$ term.  Then, if I act on the system with an external force, the system will have a reaction to that force and we see this reaction as the solution $x_p$.  In total, the system will then behave in an additive way.  The dynamics we see simply come from adding together the system behavior along with the system response.

        How do we know that this $x=x_h+x_p$ is a solution? We can check by taking
        \begin{align*}
            x''+bx'+cx&=(x_h+x_p)''+b(x_h+x_p)'+c(x_h+x_p)\\
            &= \underbrace{(x_h''+bx_h'+cx_h)}_{=0}+\underbrace{(x_p''+bx_p'+cx_p)}_{=F(t)}\\
            &=F(t).
        \end{align*}
        So the sum of a particular integral and a homogeneous solution is also a solution to the inhomogeneous equation! Now, let's see an example.

        \begin{ex}{Inhomogeneous Linear Equation with Constant Force}{inhom_linear}
        If we assume our equation has a quadratic forcing term $F(t)=2t^2$, then we have the equation
        \[
        x''+3x'+2x=2t^2.
        \]
        First, we find the solution to the homogeneous equation
        \[
        x_h''+3x_h'+2x_h=0
        \]
        which has the characteristic polynomial
        \[
        \lambda^2+3\lambda+2=0.
        \]
        The roots are then $\lambda_1=-1$ and $\lambda_2=-2.$ So we have that
        \[
        x_h(t)=C_1e^{-t}+C_2e^{-2t}.
        \]
        Next, we take an ansatz of $x_p(t)=a_0+a_1t+a_2t^2$ and call the $a_i$ coefficients the \boldgreen{undetermined coefficients}.  Then we have
        \[
        x_p'=a_1+2a_2t \qquad \textrm{and} \qquad x_p''=2a_2.
        \]
        We can plug these into our inhomogeneous equation
        \begin{align*}
        x''+3x'+2x&= 2a_2+3(a_1+2a_2t)+2(a_0+a_1t+a_2t^2)
        \end{align*}
        which we know should also be equal to the forcing term $F(t)$.  This gives us the equation
        \[
        2a_2+3(a_1+2a_2t)+2(a_0+a_1t+a_2t^2)=2t^2.
        \]
        We rearrange the terms as follows
        \[
        (2a_0+3a_1+2a_2)+(2a_1+6a_2)t+(2a_2)t^2=0+0t+2t^2
        \]
        which gives us the system of equations
        \begin{align*}
            2a_0+3a_1+2a_2&=0\\
            2a_1+6a_2&=0\\
            2a_2&=2.
        \end{align*}
        We can solve these to find $a_2=1$, $a_1=-3$ and $a_0=\frac{7}{2}$. So the particular integral is $x_p(t)=t^2-3t+\frac{7}{2}$.  Thus we have a solution
        \[
        x=x_h+x_p=C_1e^{-t}+C_2e^{-2t}+t^2-3t+\frac{7}{2}.
        \]
        If we were given initial data, we could solve for $C_1$ and $C_2$.
        \end{ex}


        \noindent Here's a table of the forms of $F(t)$ which will be solvable. You can use this to find an ansatz for the particular integral.  Once you know the ansatz it comes down to solving for the coefficients as we did in the previous example.
        \begin{table}[H]
        \centering
        \renewcommand{\arraystretch}{1.75}
            \begin{tabular}{c|c}
                Forcing Term $F(t)$&  Particular Integral $x_p$\\
                \hline
                $ke^{at}$ & $Ce^{at}$\\
                \hline
                $kt^n$ ~ $(n=0,1,2,\dots)$ & $\sum_{j=0}^n a_j t^j$\\
                \hline
                $k\cos(at)$ ~\textrm{or}~ $k\sin(at)$ & $K\cos(at)+M\sin(at)$\\
                \hline
                $ke^{at}\cos(bt)$ ~\textrm{or}~ $ke^{at}\sin(bt)$ & $e^{at}(K\cos(bt)+M\sin(bt))$\\
                \hline
                $\displaystyle{\left(\sum_{j=0}^n k_jt^j\right) \cos(bt)}$ ~\textrm{or}~ $\displaystyle{\left(\sum_{j=0}^n k_jt^j\right) \sin(bt)}$ & $\displaystyle{\sum_{j=0}^n\left( Q_jt^j \cos(bt) + R_jt^j \sin(bt)\right)}$\\
                \hline
                $\displaystyle{\left(\sum_{j=0}^n k_jt^j\right) e^{at}\cos(bt)}$ ~\textrm{or}~ $\displaystyle{\left(\sum_{j=0}^n k_jt^j\right) e^{at}\sin(bt)}$ & $\displaystyle{e^{at}\left(\sum_{j=0}^n \left( Q_jt^j \cos(bt) + R_jt^j \sin(bt)\right)\right)}$
            \end{tabular}
    \end{table}



        \newpage
        \section*{Problems}
        \todo{These problems could be ordered better and, again, probably laid out in sections instead.}
            \begin{problem}
        Complete the exercises throughout the chapter.
        \end{problem}

\begin{problem}
    What is an (ordinary) differential equation? Explain what it means to be a general and particular solution to a differential equation.
\end{problem}

\begin{problem}
    Look up an ordinary differential equation in chemistry that interests you.  Write it down, and explain what it attempts to model. Why does it interest you?
\end{problem}

\begin{center}
    Problems 4.4-4.9 are related.
\end{center}

\begin{problem}
    (Newton's law of cooling) Write down a differential equation that models the following scenario:\\

    \noindent\emph{The temperature of a substance in an ambient environment changes temperature over time proportionally to the difference of the temperature of the substance from the temperature of the ambient environment. Assume that the ambient environment is large enough to maintain a constant temperature.}\\

    \noindent Let $T(t)$ be the temperature of the substance, $T_a$ be the ambient temperature, and $k$ be the constant of proportionality.
\end{problem}

\begin{problem}
    With the equation found above, find a general solution.
\end{problem}

\begin{problem}
    With the parameter values $T_a=100$, $k=1$, and initial data $T(0)=50$, find the particular solution.  Find as well the particular solution when $T(0)=55$ and when $T(0)=150$. Plot each of the particular solutions and explain the results.
\end{problem}


\begin{problem}
    What happens instead if the initial temperature is equal to the ambient temperature? That is, when $T(0)=T_a=100$. Does your solution reflect this? Does this make physical sense? Explain.
\end{problem}

\begin{problem}
    Let $\delta = (T_a-T)$. Show that the equation you found in Problem 1 reduces to
    \[
    \delta' = -k\delta.
    \]
    What is this equation describing physically? Explain.
\end{problem}

\begin{problem}
    The equation you arrived at earlier, $\delta'(t) = -k\delta(t)$, is \emph{autonomous}.  In this particular instance, it means that this problem has a derivative that is independent of time $t$.  In fact, for this system, this essentially means that total energy is conserved!
    \begin{enumerate}[(a)]
        \item Using the slope field generator found at: \url{https://www.desmos.com/calculator/p7vd3cdmei}, plot the slope field in the $t\delta$-plane and explore what happens as you vary $k$. What happens when $k=0$? How about $k>0$? How about $k<0$?
        \item In this slope field plot, explain the symmetry.  Can you see why this shows that we can always choose the initial time to be $t_0=0$ regardless of the value of the initial condition?
        \item $k$ represents the conductivity of the object.  Explain what the the solutions for $k<0$, $k=0$, and $k>0$ mean physically. Should we think of objects with $k\leq 0$?
        \item If instead we had an equation $y'=-kty$, can you see why we can no longer simply choose $t_0=0$ as the initial time?
    \end{enumerate}
\end{problem}

\begin{center}
    Problems 4.10-4.12 are related.
\end{center}

\begin{problem}
	\textbf{(3 pts.)} (Hooke's law) Write down an initial value problem based on the following statement.\\
	\emph{``The rate of change of the rate of change of position of a mass is proportional to the position but in the opposite direction."}
\end{problem}

\begin{problem}
    Show that $x=c_1\sin(t)+c_2\cos(t)$ is a general solution to the equation
    \[
        x''+x=0.
    \]
\end{problem}

\begin{problem}
    Find the particular solution if $x(0)=1$ and $x'(0)=0$. Plot your solution in the $t,x$-plane and in the $x,x'$-plane.
\end{problem}



\begin{problem}
    Consider the differential equation
    \[
    x'=\frac{x^2+tx+t^2}{tx}.
    \]
    \begin{enumerate}[(a)]
        \item Let $f(x,t)=\frac{x^2+tx+t^2}{tx}$.  Show that $f(\lambda x, \lambda t)=f(x,t)$.
        \item Use the substitution $u=\frac{x}{t}$ in order to make the original equation separable.
        \item Find the general solution to this separable equation in terms of $u$ and $t$. You may use Wolfram Alpha to compute the necessary integral.
        \item Find the solution to the original equation using the substitution $u=\frac{x}{t}$ and your solution from (c).
    \end{enumerate}
\end{problem}

\begin{problem}
(Characteristic polynomial) Consider the following (differential) equation
\[
ax''(t)+bx'(t)+cx(t)=0.
\]
This equation can be converted to a quadratic equation
\[
a\lambda^2 + b\lambda + c = 0.
\]
What are the roots to this equation?
\end{problem}

\begin{problem}
(Why $i$?) Consider the following (differential) equation
\[
x''(t)=-x(t).
\]
Now, let $x(t)=e^{it}$.  Show that the above expression is true. (\emph{Note: this is the same equation as in Problem 4.10, can you see why this must be true using Euler's formula?})
\end{problem}

\begin{problem}
Objects near Earth fall due to the force of gravity.  The acceleration of an object due to gravity (regardles of mass) is then
\[
y''=g,
\]
where $y(t)$ represents the height above the ground at time $t$ and $g\approx -9.8\frac{m}{s^2}$ is constant.
\begin{enumerate}[(a)]
    \item Find the general solution to the equation.
    \item Given the initial data $y(0)=0$ and $y'(0)=1$, find the particular solution.
    \item At what time $t>0$ does the object first contact the ground?
    \item Plot your solution only over the range of time that makes physical sense.
\end{enumerate}
\end{problem}

\begin{problem}
Consider the autonomous (Hamilton) equation
\[
	x'=ix
\]
with initial condition $x(0)=1$. \emph{Hint: Think about what this is saying -- the velocity of the particle $x'$ is a rotation by $\pi/2$ of the position $x$.}
\begin{enumerate}[(a)]
	\item Find the particular solution to this equation.
	\item Plot the function $x(t)$ in the complex plane.
	\item Show that your solution $x(t)$ solves the ODE
	\[
		x'' = - x
	\]
	which is seen in Problems 4.10 and 4.14. \emph{It is worth thinking about this too -- the acceleration of a particle $x''$ is in the opposite direction of the position $x$, i.e., centripetal motion.}
\end{enumerate}
\end{problem}

\begin{problem}
Consider the following autonomous equation.
    \[
    x'=-x^2.
    \]
    \begin{enumerate}[(a)]
        \item Draw the phase line for this system. What are the equilibrium point(s)? Which equilibria are stable? Which are unstable? Explain.
        \item Find a general solution to the ODE.
        \item Explain how your general solution fits the qualitative behavior expected from the phase line. That is, can you show that limits of your general solution match your qualitative analysis?
        \item Can $x(0)=0$ be an initial condition? Explain. \emph{Hint: your analysis from the phase line may prove to be more useful than the general solution you found.}
    \end{enumerate}
\end{problem}


\begin{problem}
Let $y$ be a function of $x$ and consider the following differential equation.
\[
y' = y\cos(x).
\]
\begin{enumerate}[(a)]
    \item What is the order of this equation? Is the equation separable? Explain.
    \item Plot an approximation of the slope field for this equation using this Desmos link: \url{https://www.desmos.com/calculator/e93gktwtfo}. \emph{Note that you will have to modify the $g(x,y)$ equation in that page.}
    \item Find the general solution to this equation.
    \item Given the initial data $y(0)=1$, find the particular solution.
    \item Plot this function over your slope field. Explain how you could have approximated this solution using just the slope field.
    \item Explain in words what the solution describes if we let $y(x)$ be the position of some object and $x$ represents time.
\end{enumerate}
\end{problem}

\begin{problem}
Consider the differential equation
\[
x'=\frac{x+t}{t}.
\]
\begin{enumerate}[(a)]
    \item Let $f(x,t)=\frac{x+t}{t}$. Show that $f(x,t)=f(\lambda x, \lambda t)$.
    \item Given (a) holds, use the change of variables $u=\frac{x}{t}$ to rewrite the differential equation as a separable equation in terms of $u$.
    \item Find the general solution to the equation and write your solution in terms of the original variables $t$ and $x$.
\end{enumerate}
\end{problem}

\begin{problem}
Find the general solution to the following equation.
\[
tx'+2x=\frac{\sin(t)}{t}.
\]
Show that your solution is correct. (\emph{Hint: can you use an integrating factor?})
\end{problem}

\begin{problem}
Write down the equations for each of the reactants and products for the following reactions.
\begin{enumerate}[(a)]
    \item $A + 3B + C \xrightarrow{k} 2D+2E$.
    \item $A \xrightarrow{k_1} B + C \xrightarrow{k_2} D$.
\end{enumerate}
\end{problem}

\begin{problem}
Consider the following reaction
\[
A \xrightarrow{k_1} B \xrightarrow{k_2} C.
\]
For the following parts, use the link: \url{https://www.desmos.com/calculator/srrpeadlou}.
\begin{enumerate}[(a)]
    \item Compare and contrast the reactions that take place given the three different scenarios for initial conditions. Explain why what the graph displays makes sense and include your graphs.
    \begin{itemize}
        \item $[A]_0 = 1$, $[B]_0=0$, and $[C]_0 =0$.
        \item $[A]_0 = 0$, $[B]_0=1$, and $[C]_0 =0$.
        \item $[A]_0 = 0$, $[B]_0=0$, and $[C]_0 =1$.
    \end{itemize}
    \item For the initial conditions $[A]_0 = 1$, $[B]_0=0$, and $[C]_0 =0$, explain what happens when you let
    \begin{itemize}
        \item $k_1=0$ and $k_2=1$,
        \item $k_1=1$ and $k_2=0$.
    \end{itemize}
    Include plots for these cases as well.
    \item Consider the initial conditions $[A]_0 = 1$, $[B]_0=0$, and $[C]_0 =0$ and rate constants $k_1=1$ and $k_2=2$. Then, choose initial conditions of your own and compare your plots with the other initial conditions. Why do yours behave the way they do? Include your plots.
\end{enumerate}
\end{problem}

\begin{problem}
Consider the second order chemical reaction given by
\[
A+B \xrightarrow{k} \textrm{Products}.
\]
\begin{enumerate}[(a)]
    \item Write a \emph{system} of differential equations to describe the concentration of the reactants $A$ and $B$ (this means write one for each).
    \item The concentrations of $A$ and $B$ can be related to each other in the following way: Let $A=A_0-x$ and $B=B_0-x$. Here, we think of $x$ as the amount of each chemical that has reacted, and note that it depends on time $t$. Use this change of variables to rewrite the differential equation for chemical $A$ in terms of $x$ and $t$.
    \item Solve the differential equation in (b) with the initial condition $x(0)=0$.You will need to use \emph{partial fraction decomposition} to evaluate the integral.
\end{enumerate}
\end{problem}

\begin{problem}
If $x_1(t)$ and $x_2(t)$ are solutions to the differential equation
\[
x'' + bx' +cx = 0
\]
is $x=x_1+x_2+k$ for a constant $k$ always a solution? Is the function $y=tx_1$ a solution? Explain.
\end{problem}


\begin{problem}
Consider the following initial value problem:
\begin{align*}
    x''+4x'+3x&=0
\end{align*}
with initial data $x(0)=1$, $x'(0)=0$.
\begin{enumerate}[(a)]
    \item Find the solution.
    \item Sketch a plot of the solution.
    \item Explain in words what is happening to the solution as time goes on. What happens as $t\to \infty$?
\end{enumerate}
\end{problem}

\begin{problem}
Write down a homogeneous second-order linear differential equation where the system displays a decaying oscillation.
\end{problem}

\begin{problem}
Consider the following differential equation:
\[
x''+2x'+x=\sin(t)
\]
\begin{enumerate}[(a)]
    \item Find the homogeneous solution $x_H(t)$.
    \item Find the particular integral $x_P(t)$.
    \item Find the specific solution corresponding to the initial data $x(0)=0$, $x'(0)=0$.
    \item Plot the curve $(x(t),x'(t))$ in the plane (we often call this \emph{phase space}). Use this link here: \url{https://www.desmos.com/calculator/ouqwcxj2xz}. Use the time range $t\in [0,10\pi]$.
    \item Describe what happens with this system over time. Does it seem to approach some kind of stable solution? Note that this stable solution could be periodic.
\end{enumerate}
\end{problem}

\begin{problem}
    Let's revisit Newton's law of cooling and describe the equilibria of the system.
\begin{enumerate}[(a)]
    \item Is the system autonomous? Explain.
    \item Draw a phase line for the system.
    \item Can you explain why the value you find for the equilibrium makes sense? Is this equilibrium be stable? Does this make sense? Explain.
\end{enumerate}
\end{problem}

\begin{problem}
Are the following ODEs separable, autonomous, linear, nonlinear, or none of the above? Keep in mind that some may satisfy more than one property!
    \begin{enumerate}[(a)]
        \item $x' = \sin(tx)$.
        \item $x' = \sin(x)$.
        \item $x' = t^2 x$.
        \item $e^t x' + tx = e^{-k t} \cos(\omega t).$
    \end{enumerate}
\end{problem}

\begin{problem}
For the following problems, show that the equation is linear by writing the equation in a recognizable form.
    \begin{enumerate}[(a)]
        \item $\frac{x'}{x} = t$.
        \item $2xx'+x^2=xt$.
        \item $\tan(t) x' + \sin(t) x = \ln(t).$
    \end{enumerate}
\end{problem}

\begin{problem}
    For the above linear equations, determine the integrating factor (even if you cannot compute the integral) and determine the solution $x(t)$ (again, even if you cannot compute the integral).
\end{problem}

\begin{problem}
    Find the solution to the equation $x'+x=t^2$.
\end{problem}

\begin{problem}
    Consider the dissociation chemical reaction
    \[
        AB \to A + B.
    \]
    Write down an equation for each species $A$, $B$, and $AB$.  Find a solution.
\end{problem}

\begin{problem}
    Compare and contrast the equations for the above reaction and the synthesis reaction
    \[
        A+B \to AB.
    \]
    \emph{Hint: maybe this is a bit of an odd way to think, but is one just the reversal in time of the other?}
\end{problem}

\begin{problem}
    Photosynthesis is an extremely important chemical reaction where plants convert carbon dioxide and water into glucose and oxygen. That is,
    \[
        6CO_2 + 6H_2O \to 6C_6 H_{12} O_6 + 6 O_2.
    \]
    \begin{enumerate}[(a)]
        \item Write down the equations that describe the above equation.
        \item Do we have any techniques to solve this type of equation (yet)?
        \item The reaction also depends on the intensity of light.  How can we implement this into the equation?
    \end{enumerate}
\end{problem}

\begin{problem}
Everybody loves combustion. It keeps us warm and it looks cool on the 4$^\textrm{th}$ of July!  Anyhow, combustion of propane in your grill at home is given by the equation
    \[
        C_3 H_8 + 5O_2 \to 4H_2O + 3CO_2 + \textrm{Energy}.
    \]
    \begin{enumerate}[(a)]
        \item Write down the equations describing the above reaction (not including the energy term).
        \item If a specific amount of energy is given off by each reaction (in this case, $2043\textrm{kJ}$ of energy per mole), include an equation for the total energy created if we begin with $1\textrm{kg}$ of propane in an abundance of oxygen.
        \item Can you give some analogy to how much energy this is so that we may understand the usefulness a bit more? E.g., how many hot dogs could I cook on the 4$^\textrm{th}$ of July?
    \end{enumerate}

\end{problem}

\begin{problem}
Write down the characteristic polynomial for the following equations.  Then, find the roots to the characteristic polynomial and write down the general solution.
\begin{enumerate}[(a)]
    \item $x''+x'+x=0$.
    \item $x''-x'-x=0$.
    \item $x''-x'+x=0$.
    \item $x''+x'-x=0$.
\end{enumerate}
\end{problem}

\begin{problem}
For the above solutions, analyze their behavior qualitatively. That is, do the solutions oscillate, grow, decay, or some combination of these, or something else entirely?
\end{problem}

\begin{problem}
Consider the equation
\[
x''+bx'+cx=0.
\]
The roots to the characteristic polynomial are then
\[
\lambda = \frac{-b\pm \sqrt{b^2-4c}}{2}.
\]
\begin{enumerate}[(a)]
    \item Explain why if $c>0$ and $b=0$ the solution $x(t)$ will be purely oscillatory.
    \item Explain why if $b>0$ and $b^2<4c$, the solution will oscillate and decay.
    \item Explain why if $b<0$ and $b^2<4c$, the solution will oscillate and grow.
\end{enumerate}
\end{problem}

\begin{problem}
Write down a second order linear differential equation that oscillates and also decays over time.
\end{problem}

\begin{problem}
Consider the following differential equation
\[
x''+x=0.
\]
\begin{enumerate}[(a)]
    \item Find the general solution to this equation.
    \item Given the initial conditions $x(0)=1$ and $x'(0)=1$, find the particular solution.
    \item Plot your particular solution.
    \item Does the solution grow or decay over time?
    \item What is $\lim_{t\to \infty}x(t)$?
\end{enumerate}

\end{problem}

\begin{problem}
Next, consider a related equation
\[
x''+x=t.
\]
that has an additional linear external force.
\begin{enumerate}[(a)]
    \item What is the solution to the homogenous equation?
    \item Find the particular integral with the given forcing term.
    \item What is the specific solution to this equation?
    \item Does the solution grow or decay over time?
    \item What is $\lim_{t\to \infty}x(t)$?
\end{enumerate}
\end{problem}

\begin{problem}
Consider now the equation
\[
x''+x=F(t)
\]
where the external force is $F(t)=\cos(t)$.
\begin{enumerate}[(a)]
    \item Find the particular integral with the given forcing term.
    \item What is the specific solution to this equation?
    \item What is $\lim_{t\to \infty}$? What does this mean about the growth or decay of the solution over time?
\end{enumerate}
\end{problem}

\begin{problem}
Consider the endothermic breakdown of a molecule $x$ given by
\[
x \xrightarrow{kt^2} \textrm{Products}
\]
where we let $x(t)$ denote the concentration of reactants. Since the reaction is endothermic, if we also heat up the solution over time, we get a factor of $t^2$ as well since the reaction occurs more readily in higher temperatures. The concentration decreases over time based on differential equation
\[
x'=-kt^2(x-x_e).
\]
where $x_e$ is a constant that denotes the equilibrium concentration.
\begin{enumerate}[(a)]
    \item Write an equivalent equation with the change of variables $\delta=x-x_e$.
    \item Find the general solution to this new equation.
    \item What is the general solution in terms of the original variables $x$?
    \item Given the initial amount of $x$ is $x(0)=1$, the equilibrium concentration is $x_e=1/2$, and $k=1$, find the particular solution for $x(t)$.
    \item Does this reaction ever reach the equilibrium state?
\end{enumerate}
\end{problem}

\begin{problem}
Let the height above ground at time $t$ be given by the function $y(t)$. A ball falling through air experiences gravitational acceleration and damping due to air friction. It follows the differential equation
\[
y'' = -ky' - g.
\]
\begin{enumerate}[(a)]
    \item Describe the type of this equation (e.g., separable, autonomous, linear, etc.). What is the order?
    \item Find the solution to the homogeneous equation. Call this solution $x_H$.
    \item Find the particular integral to the inhomogeneous equation. Call this $x_P$.
    \item What is the general solution to the ODE given in this problem?
    \item Suppose that $k=1$ and $g=10$. Let $y(0)=1,000$ and $y'(0)=0$. Find the particular solution to this problem.
    \item Plot your particular solution. Plot the derivative of your solution $y'(t)$ as well. If your solution is correct, the derivative (velocity) $y'$ should approach a constant value called the \emph{terminal velocity}.
\end{enumerate}
\end{problem}

\begin{problem}
The spring/mass harmonic oscillator is given by the equation
\[
x''+ \frac{k}{m}x = 0.
\]
where $m$ is the mass of the oscillating object and $k$ is the spring constant.
\begin{enumerate}[(a)]
    \item Show that
    \[
    x(t)=A \cos\left(\sqrt{\frac{k}{m}}t\right).
    \]
    solves the initial problem with initial data $x(0)=A$ and $x'(0)=0$.
    \vspace*{6cm}
    \item Since this system has no damping, the total energy is \underline{conserved at all times}.  In particular, the total energy is given by
    \[
    E=\frac{1}{2}mx'(t)^2+\frac{1}{2}kx(t)^2 = \textrm{constant}.
    \]
    and is equal for all times $t\geq 0$. What is the total energy of the given particular solution $x(t)=A\cos\left(\sqrt{\frac{k}{m}}t\right)$?
    \vspace*{5cm}

\end{enumerate}
\end{problem}

\begin{problem}
Consider the inhomogeneous linear differential equation
\[
x''+\omega^2x=\cos(\omega t),
\]
where $x(t)$ is a function of $t$. This is an example of resonance.
\begin{enumerate}[(a)]
    \item Find the general homogeneous solution $x_h(t)$.
    \item Show that your solution solves the homogeneous equation
    \[
    x_h''+\omega^2x_h=0.
    \]
    \item For the particular integral, $x_p(t)$, take the ansatz
    \[
    x_p(t)=C_1t\cos(\omega t) + C_2 t\sin(\omega t)
    \]
    and find the undetermined coefficients $C_1$ and $C_2$.
    \item Show that $x=x_h+x_p$ solves the inhomogeneous equation
    \[
    x''+\omega^2 x = \cos(\omega t).
    \]
\end{enumerate}
\end{problem}





